\documentclass[]{article}
\usepackage{lmodern}
\usepackage{amssymb,amsmath}
\usepackage{ifxetex,ifluatex}
\usepackage{fixltx2e} % provides \textsubscript
\ifnum 0\ifxetex 1\fi\ifluatex 1\fi=0 % if pdftex
  \usepackage[T1]{fontenc}
  \usepackage[utf8]{inputenc}
\else % if luatex or xelatex
  \ifxetex
    \usepackage{mathspec}
  \else
    \usepackage{fontspec}
  \fi
  \defaultfontfeatures{Ligatures=TeX,Scale=MatchLowercase}
\fi
% use upquote if available, for straight quotes in verbatim environments
\IfFileExists{upquote.sty}{\usepackage{upquote}}{}
% use microtype if available
\IfFileExists{microtype.sty}{%
\usepackage{microtype}
\UseMicrotypeSet[protrusion]{basicmath} % disable protrusion for tt fonts
}{}
\usepackage[margin=1in]{geometry}
\usepackage{hyperref}
\hypersetup{unicode=true,
            pdftitle={Phylogeny: An introductory R Notebook by T. Hollis},
            pdfborder={0 0 0},
            breaklinks=true}
\urlstyle{same}  % don't use monospace font for urls
\usepackage{graphicx,grffile}
\makeatletter
\def\maxwidth{\ifdim\Gin@nat@width>\linewidth\linewidth\else\Gin@nat@width\fi}
\def\maxheight{\ifdim\Gin@nat@height>\textheight\textheight\else\Gin@nat@height\fi}
\makeatother
% Scale images if necessary, so that they will not overflow the page
% margins by default, and it is still possible to overwrite the defaults
% using explicit options in \includegraphics[width, height, ...]{}
\setkeys{Gin}{width=\maxwidth,height=\maxheight,keepaspectratio}
\IfFileExists{parskip.sty}{%
\usepackage{parskip}
}{% else
\setlength{\parindent}{0pt}
\setlength{\parskip}{6pt plus 2pt minus 1pt}
}
\setlength{\emergencystretch}{3em}  % prevent overfull lines
\providecommand{\tightlist}{%
  \setlength{\itemsep}{0pt}\setlength{\parskip}{0pt}}
\setcounter{secnumdepth}{0}
% Redefines (sub)paragraphs to behave more like sections
\ifx\paragraph\undefined\else
\let\oldparagraph\paragraph
\renewcommand{\paragraph}[1]{\oldparagraph{#1}\mbox{}}
\fi
\ifx\subparagraph\undefined\else
\let\oldsubparagraph\subparagraph
\renewcommand{\subparagraph}[1]{\oldsubparagraph{#1}\mbox{}}
\fi

%%% Use protect on footnotes to avoid problems with footnotes in titles
\let\rmarkdownfootnote\footnote%
\def\footnote{\protect\rmarkdownfootnote}

%%% Change title format to be more compact
\usepackage{titling}

% Create subtitle command for use in maketitle
\newcommand{\subtitle}[1]{
  \posttitle{
    \begin{center}\large#1\end{center}
    }
}

\setlength{\droptitle}{-2em}

  \title{Phylogeny: An introductory R Notebook by T. Hollis}
    \pretitle{\vspace{\droptitle}\centering\huge}
  \posttitle{\par}
    \author{}
    \preauthor{}\postauthor{}
    \date{}
    \predate{}\postdate{}
  

\begin{document}
\maketitle

\section{1. Introduction}\label{introduction}

This is an \href{http://rmarkdown.rstudio.com}{R Markdown} Notebook.
When you execute code within the notebook, the results appear beneath
the code. I decided to write up my oral examination into an R notebook
to engage in literate programming (as mentionned in a bonus learning
unit that I took).

Thomas Hollis (BCH441, University of Toronto) -v2.1

\begin{itemize}
\item
  Purpose: Phylogeny Oral Evaluation (build \& analyse aphylogenetic
  tree)
\item
  Bugs \& issues: no bugs, no issues, no warnings
\item
  Acknowledgements: thanks to Prof.~Steipe's learning units on Phylogeny
  which were of great help
\end{itemize}

Disclaimer: I used a patch designed by myself (see mailing list) in
earlier units so I hope this will not hinder the oral test.

\section{2. Code}\label{code}

\subsection{2.1. Import required
libraries}\label{import-required-libraries}

As always, lets import our required libraries and data before starting:

\subsection{2.2. Data Preparation}\label{data-preparation}

\subsubsection{2.2.1 Alignment}\label{alignment}

Let's first align some sequences (all sequences in the database +
KILA\_ESSCO for rooting the tree):

\subsubsection{2.2.2 Get the sequence of the SACCE APSES
domain}\label{get-the-sequence-of-the-sacce-apses-domain}

\subsubsection{2.2.3 Extract the APSES domains from the
MSA}\label{extract-the-apses-domains-from-the-msa}

It is worth noting that this includes PSI-BLAST results which can be
found in MYSPE\_APSES\_PSI-BLAST.json. It addition, APSESmsa is of type
``AAStringSet'' not ``MsaAAMultipleAlignment'' as seen before.

\subsubsection{2.2.4 Process the APSESmsa data for Tree
Building}\label{process-the-apsesmsa-data-for-tree-building}

We need to mask some of the collumns. To do this we must first convert
to a matrix of characters, then mask, then convert back and export to
multi-FASTA.

\subsection{2.3 Build the tree using
PROML}\label{build-the-tree-using-proml}

Import the clean, aligned \& masked data. Use this to build the tree.
(WARNING: took around 8h to run)

\subsection{2.4 Analyse the tree}\label{analyse-the-tree}


\end{document}
